\documentclass[a4paper]{article}
\usepackage[utf8]{inputenc}
\usepackage[T1]{fontenc}
\usepackage[top=5mm, bottom=5mm, left=5mm, right=5mm]{geometry}
\usepackage[hidelinks]{hyperref}
\usepackage{PTSansNarrow,tikz}
\renewcommand*\familydefault{\sfdefault}
\usepackage[spanish]{babel}
\usepackage{sty/portada}

\usetikzlibrary{shapes.multipart}

%Options for timetable contents
\def\firsthour{8}
\def\lasthour{21}
\def\daynames{lunes, martes, miércoles, jueves, viernes}

%Options for timetable drawing
\def\daywidth{3.8cm}   %approx \textwidth / 6
\def\hourheight{2cm} %approx \textheight / (\lasthour - \firsthour + 1)

\title{Horarios de clase para el curso 2019--2020}
\author{}

\begin{document}

\portada
\clearpage\restoregeometry

\newgeometry{%
    left=3cm,right=3cm,
    top=5cm, bottom=5cm}
\tableofcontents
\clearpage

\section*{Prefacio}

\noindent En este cuadernillo se incluyen horarios generados automáticamente a partir de los datos de la \textbf{Intranet de la Escuela}\footnote{\url{https://intranet.eii-to.uclm.es/horarios}}.  Sin embargo, no incluye información sobre la periodicidad de cada una de las clases ni la distribución en grupos.

Algunos laboratorios se realizan de forma concentrada en una única semana; otras clases se imparten en un periodo de cinco, siete, u once semanas; otras clases se repiten en semanas alternas o se distribuyen en varios grupos.  Dada la variedad de casos hemos decidido simplificar al máximo la versión en PDF, incluyendo exclusivamente información de asignatura, profesores y aula.  Utiliza la \textbf{Intranet de la Escuela} para ampliar la información.

\clearpage\restoregeometry

\setcounter{section}{1}
\setcounter{subsection}{1}
\setcounter{subsubsection}{0}
\phantomsection
\addcontentsline{toc}{section}{Ingeniería Eléctrica}
\phantomsection
\addcontentsline{toc}{subsection}{Primer semestre}

\input{Primer_semestre_IE_01.tex}
\input{Primer_semestre_IE_02.tex}
\input{Primer_semestre_IE_03.tex}
\input{Primer_semestre_IE_04.tex}

\setcounter{subsection}{2}
\setcounter{subsubsection}{0}
\phantomsection
\addcontentsline{toc}{subsection}{Segundo semestre}

\input{Segundo_semestre_IE_01.tex}
\input{Segundo_semestre_IE_02.tex}
\input{Segundo_semestre_IE_03.tex}
\input{Segundo_semestre_IE_04.tex}

\setcounter{section}{2}
\setcounter{subsection}{1}
\setcounter{subsubsection}{0}
\phantomsection
\addcontentsline{toc}{section}{Ingeniería Electrónica Industrial y Automática}
\phantomsection
\addcontentsline{toc}{subsection}{Primer semestre}

\input{Primer_semestre_IEIA_01.tex}
\input{Primer_semestre_IEIA_02.tex}
\input{Primer_semestre_IEIA_03.tex}
\input{Primer_semestre_IEIA_04.tex}

\setcounter{subsection}{2}
\setcounter{subsubsection}{0}
\phantomsection
\addcontentsline{toc}{subsection}{Segundo semestre}

\input{Segundo_semestre_IEIA_01.tex}
\input{Segundo_semestre_IEIA_02.tex}
\input{Segundo_semestre_IEIA_03.tex}
\input{Segundo_semestre_IEIA_04.tex}

\setcounter{section}{3}
\setcounter{subsection}{1}
\setcounter{subsubsection}{0}
\phantomsection
\addcontentsline{toc}{section}{Ingeniería Aeroespacial}
\phantomsection
\addcontentsline{toc}{subsection}{Primer semestre}

\input{Primer_semestre_IA_01.tex}
\input{Primer_semestre_IA_02.tex}

\setcounter{subsection}{2}
\setcounter{subsubsection}{0}
\phantomsection
\addcontentsline{toc}{subsection}{Segundo semestre}

\input{Segundo_semestre_IA_01.tex}
\input{Segundo_semestre_IA_02.tex}

\end{document}